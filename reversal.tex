\documentclass[12pt]{article}
\usepackage{graphicx}
\usepackage{caption2}
\usepackage{a4wide}
\usepackage[margin=1.5cm]{geometry}
\usepackage{natbib}
\begin{document}

\pagestyle{headings}

\def\im#1{{\mathrm{Im}(#1)}}
\def\re#1{{\mathrm{Re}(#1)}}

%\renewcommand{\baselinestretch}{2}

%\mainmatter
%\renewcommand{\captionfont}{\footnotesize}
%\setlength{\captionmargin}{\textwidth}

\title{Reversal learning in a robot food discrimination task inspired by the limbic system}

\author{Adedoyin Maria Thompson }

%\and Bernd Porr\inst{1} \and
%Florentin W\"org\"otter\inst{2}}
%\institute{Department of Electronics \&  
%Electrical Engineering, University of Glasgow,\\
%Glasgow, G12 8LT,
%Scotland, United Kingdom
%\email{\{mariat,b.porr\}@elec.gla.ac.uk}
%\and
%Bernstein Center of Computational Neuroscience,
%University G\"ottingen, Germany,
%\email{worgott@chaos.gwdg.de}}
%\author{Maria Thompson\footnote{Department of Electronics \& Electrical Engineering,
%University of Glasgow,
%Glasgow, GT12 8LT, UK,
%\texttt{mariat@elec.gla.ac.uk}}}


\maketitle

\begin{abstract}
This paper demonstrates how a model of the limbic system can be used to
perform a reversal learning task. 


\section{Introduction}
An example of reversal learning can be demonstrated by a rat. If the rat
has to learn an association of finding food in the left
compartment of a cage. After a while, the food is moved to the right
compartment of the cage. The rat has to unlearn the original association
of the food to the left hand corner of the cage so as to learn the new
association.
This reversal learning task is believed to be carried out in one of the
oldest regions of the brain, the limbic system which is responsible for its role in
mediation and reward.
The nucleus accumbens (Nacc) as well as the ventral tegmental area (VTA)
are nuclei in the limbic system. The Nucleus accumbens is a
structure in the rostrobasal forebrain and is a highly differentiated
striatal part of the ventrastraitopallidal system(Zahm 2000). Studies of
the NAcc show that it comprises of three sub territories namely, the
core, the shell and the rostral pole[Heimer 1987].

\section{The Biology}
 The core is believed
to be involved in the control of adaptive motor actions, while the
shell associates place cells activity with real rewards [Wiener 2003] as well as
stores the reward value for a conditioned stimuli. 
!!Place cell activity and the shell. 

The VTA which provides
the NAcc with the neurotransmitter, Dopamine is triggered  by
 dopaminergic neurons when a reward
has been encountered. 
!!plasticity in the NAcc.
[!!!!cite]. This burst of dopamine triggers plasticity in the Nucleus
 accumbens as well as ensures that the synaptic weights do not drift. 
Dopaminergic activity triggering plasticity in the core ensures learning
of a sensor-motor association. In order to unlearn this sensor-motor
 association in the core, a ``disappointment'' is generated by the
 shell.  

\section{The model}
The model of the limbic system comprises of the Nucleus accumbens (NAc)
shell and core, the Lateral hypothalamus (LH), the Hippocampus and the Ventral tegmental area
(VTA).

\section{The Problem}
The model of the limbic system is basically still a black box.
there are currently a few problems with the model which need to be
addressed.

\subsection{11.10.06}
\begin{enumerate}
\item  The stability of the model: The seperate influences from D1 and D2
      receptors as well as the Ventral Pallidum has been inplemented.
      The overmall model is unstable with these implementations.
\item The limit on the shell is still very much a problem.
\end{enumerate}

\end{abstract}
\end{document}
